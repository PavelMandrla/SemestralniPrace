% Nejprve uvedeme tridu dokumentu s volbami
\documentclass[english,semestral]{diploma}
% Dalsi doplnujici baliky maker
\usepackage[autostyle=true,czech=quotes]{csquotes} % korektni sazba uvozovek, podpora pro balik biblatex
\usepackage[backend=biber, style=iso-numeric, alldates=iso]{biblatex} % bibliografie
\usepackage{dcolumn} % sloupce tabulky s ciselnymi hodnotami
\usepackage{subfig} % makra pro "podobrazky" a "podtabulky"
\usepackage[cpp]{diplomalst}

% Zadame pozadovane vstupy pro generovani titulnich stran.
\ThesisAuthor{Bc. Pavel Mandrla}

\ThesisSupervisor{Mgr. Ing. Michal Krumnikl, Ph.D.}

\CzechThesisTitle{Ukázka sazby kvalifikační práce}

\EnglishThesisTitle{Diploma Thesis Typesetting Demo}

\SubmissionYear{2021}

% Pokud nechceme nikomu dekovat makro zapoznamkujeme.
\Acknowledgement{Rád bych na tomto místě poděkoval všem, kteří mi s prací pomohli, protože bez nich by tato práce nevznikla.}

\CzechAbstract{Tohle je český abstrakt. TODO}

\CzechKeywords{typografie;ce}

\EnglishAbstract{This is English abstract. TODO}

\EnglishKeywords{typography; \LaTeX; master thesis}

%\AddAcronym{DVD}{Digital Versatile Disc}
%\AddAcronym{TNT}{Trinitrotoluen}
%\AddAcronym{UML}{Unified Modeling Language}
%\AddAcronym{HTML}{Hyper Text Markup Language}
%\AddAcronym{TUG}{\TeX{} Users Group}

\addbibresource{biblatex-src.bib}

% Novy druh tabulkoveho sloupce, ve kterem jsou cisla zarovnana podle desetinne carky
\newcolumntype{d}[1]{D{,}{,}{#1}}


% Zacatek dokumentu
\begin{document}

% Nechame vysazet titulni strany.
\MakeTitlePages

% A nasleduje text zaverecne prace.
\chapter{Introduction}
\label{sec:Introduction}
The 3D printing industry has experienced a giant boom in the last fifteen years.
Intensive research and inovation in this area have greatly expanded its posibilities, and 3D printing is therefore more and more often used by big industrial companies to manufacture high-tech products.

However the roots of this industry come from open source projects, driven by a passionate community of hobbyists and tinkerers, and that is greatly reflected in todays 3D printer market.
Many companies do not focus their products on large industrial subjects, but on public. That means that the price of 3D printers has gone down, while their accessibility and capability highly increased, and it is not uncomon for a household to own a 3D printer.

With these devices becoming more widespread comes a great demand for high quality 3D models.
These can be created either by using CAD software, which creates hig precision 3D model, but has a very steep learning curve, or by creating a 3D scan of an already existing object.

As the price of an industrial 3D scanner can be very high, the 3D printing community had to turn to more cost friendly alternatives.
Devices, like the Kinect from Microsoft can be used, when scanning larger object, but when it comes to smaller ones, the quality of the resulting 3D model is not sufficient.
Several open source alternatives exist for scanning small objects.
They utilize easily sourced off the shelf parts and are therefore much more accesible to public.
Whilst these solutions are not utilising as complex technologies, as their industrial counterparts, the quality of the resulting 3D prints can still be highly satisfactory.



\endinput
\input{Chapters/Solutions.tex}
% Seznam literatury
\printbibliography[title={Literatura}, heading=bibintoc]

\end{document}
