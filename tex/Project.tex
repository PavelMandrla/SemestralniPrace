% Nejprve uvedeme tridu dokumentu s volbami
\documentclass[english,semestral]{diploma}
% Dalsi doplnujici baliky maker
\usepackage[autostyle=true,czech=quotes]{csquotes} % korektni sazba uvozovek, podpora pro balik biblatex
\usepackage[backend=biber, style=iso-numeric, alldates=iso]{biblatex} % bibliografie
\usepackage{dcolumn} % sloupce tabulky s ciselnymi hodnotami
\usepackage{subfig} % makra pro "podobrazky" a "podtabulky"
\usepackage[cpp]{diplomalst}

% Zadame pozadovane vstupy pro generovani titulnich stran.
\ThesisAuthor{Bc. Pavel Mandrla}

\ThesisSupervisor{Mgr. Ing. Michal Krumnikl, Ph.D.}

\CzechThesisTitle{Ukázka sazby kvalifikační práce}

\EnglishThesisTitle{Diploma Thesis Typesetting Demo}

\SubmissionYear{2021}

\addbibresource{biblatex-src.bib}

% Novy druh tabulkoveho sloupce, ve kterem jsou cisla zarovnana podle desetinne carky
\newcolumntype{d}[1]{D{,}{,}{#1}}


% Zacatek dokumentu
\begin{document}

% Nechame vysazet titulni strany.
\MakeTitlePages

% A nasleduje text zaverecne prace.
\chapter{Introduction}
\label{sec:Introduction}
The 3D printing industry has experienced a giant boom in the last fifteen years.
Intensive research and innovation in this area have greatly expanded its possibilities, and 3D printing is therefore more and more often used by big industrial companies to manufacture high-tech products.

However the roots of this industry come from open source projects, driven by a passionate community of hobbyists and tinkerers, and that is greatly reflected in today's 3D printer market.
Many companies do not focus their products on large industrial subjects, but on public. That means that the price of 3D printers has gone down, while their accessibility and capability highly increased, and it is not uncommon for a household to own a 3D printer.

With these devices becoming more widespread comes a great demand for high quality 3D models.
These can be created either by using CAD software, which creates high precision 3D model, but has a very steep learning curve, or by creating a 3D scan of an already existing object.

As the price of an industrial 3D scanner can be very high, the 3D printing community had to turn to more cost friendly alternatives.
Devices, like the Kinect from Microsoft can be used, when scanning larger object, but when it comes to smaller ones, the quality of the resulting 3D model is not sufficient.
Several open source alternatives exist for scanning small objects.
They utilize easily sourced off the shelf parts and are therefore much more accessible to public.
Whilst these solutions are not utilizing as complex technologies, as their industrial counterparts, the quality of the resulting 3D prints can still be highly satisfactory.



\endinput
\input{Chapters/Solutions.tex}
\chapter{My contribution //TODO -> rename}
\label{sec:contribution}
I have chosen to work with the OpenScan project because of multiple reasons.
First I was wery interested in photogrammetry and its use for 3D reconstruction.
It alseo seemed to me, that the project has a lot of potential to become succesfull even within the more casual part of the 3D printing comunity, as it is easy to setup and has a simple user interface.
It alse uses the more simplistic aproach of using photogrammetry, which requires less calibration than the use of structured light, so it is very user friendly.
I also liked that the scanner does not just rotate the object along the Z axis, but allows the camera to capture the object from the top and bottom.

Where I saw room for improvement, was that the output of the scanner is not a finished 3D representation of the scanned object, such as an .stl file, but just a set of images that the user has to process himself with an additional tool, such as Meshroom.

The author of the project has since started working on his own solution to this problem and in February launched a beta of "OpenScan Cloud", which is a cloud service, that automaticly processes the scanned images and produces the resulting 3D mesh.

I have decided to use a slightly different approach, and instead of sending the scanned pictures onto a server, where the processing and asembly would be done, I would do the required calculation locally.
The only problem with this approach, is that the Raspberry Pi, which is used by the OpenScan project, is not powerful enough, to do the required processing.
Because of this reason, I've decided to replace the Raspberry Pi with a more powerful alternative, which is the Jetson Xavier NX by Nvidia.

\endinput
% Seznam literatury
\printbibliography[title={Sources}, heading=bibintoc]

\end{document}
