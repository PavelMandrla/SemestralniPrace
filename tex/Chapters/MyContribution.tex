\chapter{My contribution //TODO -> rename}
\label{sec:contribution}
I have chosen to work with the OpenScan project because of multiple reasons.
First I was wery interested in photogrammetry and its use for 3D reconstruction.
It alseo seemed to me, that the project has a lot of potential to become succesfull even within the more casual part of the 3D printing comunity, as it is easy to setup and has a simple user interface.
It alse uses the more simplistic aproach of using photogrammetry, which requires less calibration than the use of structured light, so it is very user friendly.
I also liked that the scanner does not just rotate the object along the Z axis, but allows the camera to capture the object from the top and bottom.

Where I saw room for improvement, was that the output of the scanner is not a finished 3D representation of the scanned object, such as an .stl file, but just a set of images that the user has to process himself with an additional tool, such as Meshroom.

The author of the project has since started working on his own solution to this problem and in February launched a beta of "OpenScan Cloud", which is a cloud service, that automaticly processes the scanned images and produces the resulting 3D mesh.

I have decided to use a slightly different approach, and instead of sending the scanned pictures onto a server, where the processing and asembly would be done, I would do the required calculation locally.
The only problem with this approach, is that the Raspberry Pi, which is used by the OpenScan project, is not powerful enough, to do the required processing.
Because of this reason, I've decided to replace the Raspberry Pi with a more powerful alternative, which is the Jetson Xavier NX by Nvidia.

\endinput